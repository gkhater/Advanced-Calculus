\documentclass[11pt,a4paper]{article}

%----- ENHANCED TYPOGRAPHY -----
\usepackage[utf8]{inputenc}
\usepackage[T1]{fontenc}
\usepackage{lmodern}        % clean vector font
\usepackage{microtype}      % better justification & kerning
\usepackage{palatino} 
\usepackage{braket}    % for text & math
\usepackage{mathtools}  % in the preamble

%----- PAGE LAYOUT -----
\usepackage{geometry}
\geometry{top=1in, bottom=1in, left=1in, right=1in}
\usepackage{setspace}
\onehalfspacing  % 1.5 line spacing

%----- FANCY HEADERS & FOOTERS -----
\usepackage{fancyhdr}
\pagestyle{fancy}
\fancyhf{}
\fancyhead[LO]{\small \rightmark}
\fancyhead[RO]{\small \leftmark}
\renewcommand{\headrulewidth}{0.4pt}
\renewcommand{\footrulewidth}{0pt}

% make sections feed into \leftmark/\rightmark
\renewcommand{\sectionmark}[1]{\markboth{#1}{}}
\renewcommand{\subsectionmark}[1]{\markright{#1}}

%----- SECTION NUMBERING & TOC DEPTH -----
\setcounter{secnumdepth}{3}  % number down to \subsubsection
\setcounter{tocdepth}{2}     % show ToC down to \subsection

%----- AMS MATH & THEOREM STYLES -----
\usepackage{amsmath,amssymb,mathtools}
\usepackage{amsthm}

% definitions, examples, remarks upright
\theoremstyle{definition}
\newtheorem{definition}{Definition}[section]
\newtheorem{example}[definition]{Example}
\newtheorem{remark}[definition]{Remark}

% theorems, lemmas, corollaries italic
\theoremstyle{plain}
\newtheorem{theorem}[definition]{Theorem}
\newtheorem{lemma}[definition]{Lemma}
\newtheorem{proposition}[definition]{Proposition}
\newtheorem{corollary}[definition]{Corollary}

% unnumbered proof environment
\theoremstyle{remark}

%----- OTHER PACKAGES -----
\usepackage{graphicx}
\usepackage{tikz}
\usetikzlibrary{calc, matrix, decorations.pathreplacing, positioning}
\usepackage{tikz-cd}
\usepackage{hyperref}
\hypersetup{colorlinks,
linkcolor=blue, citecolor=purple, urlcolor=teal}
\usepackage{enumitem}
\setlist[itemize]{nosep, left=1.5em}
\usepackage{booktabs}
\usepackage{listings}
\lstset{
basicstyle=\ttfamily\small,
numbers=left,
numbersep=5pt,
frame=single,
breaklines=true
}
\usepackage{xcolor}
\definecolor{shade}{HTML}{F5F5F5}
\usepackage{float}
%----- CUSTOM MACROS -----
\newcommand{\F}{\mathbb{F}}
\newcommand{\code}[1]{\texttt{#1}}
\newcommand{\dist}[2]{d\bigl(#1,#2\bigr)}
\newcommand{\R}{\mathbb{R}}
\newcommand{\Z}{\mathbb{Z}}
\newcommand{\N}{\mathbb{N}} 
\newcommand{\Q}{\mathbb{Q}} 
\newcommand{\C}{\mathbb{C}}
\renewcommand{\set}[1]{\left\{ #1 \right\}}
\newcommand{\angles}[1]{\langle #1 \rangle}
\newcommand{\abs}[1]{\lvert #1 \rvert}
\newcommand{\norm}[1]{\lVert #1 \rVert}

% \usepackage{mathtools} 
% \DeclarePairedDelimiter{\angles}{\langle}{\rangle} 
% \DeclarePairedDelimiter{\braces}{\left\{}{\right\}} 
% \DeclarePairedDelimiter{\abs}{\lvert}{\rvert} 
% \DeclarePairedDelimiter{\norm}{\lVert}{\rVert}

%----- TITLE METADATA -----
\title{\LARGE\bfseries Advanced Calculus}
\author{Georges Khater \\ \small American University of Beirut, Math 223}
\date{\today}

%===============================================
\begin{document}
\maketitle
\tableofcontents
\bigskip

\section{Metric Spaces}
\begin{definition}
    Let $X$ be a set, a \emph{metric} $d$ is a map 
    $$d \colon X \times X \to \R$$
    such that: 
    \begin{enumerate}[label = (\roman*)]
        \item $d(x,y) \geq 0$ with equality iff $x = y$. 
        \item $d(x,y) = d(y,x)$
        \item $d(x,z) \leq d(x,y) + d(y,z)$
    \end{enumerate}
\end{definition}

\begin{example}
    Given a vector space $V$ and norm $\norm{.}$ we can define the metric 
    $$d(x,y) = \norm{x-y}$$
\end{example}

\textbf{Question.} Given a vector space $V$ and a metric $d$ on $V$, does $d$ "come from" a norm? i.e does there exist some norm $\norm{}$ s.t 
$$d(x,y) = \norm{x-y}$$
The answer in general is no (consider the discrete metric, or even any bounded metric: this contradicts homogeneity). 

\paragraph{Notation.} Take $(X,d)$ metric space $\delta > 0$ and $x_0 \in X$, the \emph{$\delta$-neighborhood} of $x_0$ 
$$N_\delta(x_0) := \set{x \in X \colon d(x,x_0) < \delta}$$

\begin{definition}
    $(X,d)$ metric space and $E \subset X$ 
    \begin{enumerate}[label = (\roman*)]
        \item $x_0 \in X$ is an \emph{interior point} of $E$ iff $\exists \delta > 0$ s.t 
        $$N_\delta (x_0) \subset E$$
        \item $x_0 \in X$ is a \emph{limit point} iff $\forall \delta > 0$ 
        $$N_\delta^* (x_0) \cap E \neq \varnothing$$
        (We denote by $E'$ the set of limit points of $E$) 
        \item $E$ is \emph{open} iff every $x_0 \in E$ is an interior point of $E$. 
        \item $E$ is \emph{closed} iff every $E' \subset E$.  
        \item The \emph{closure} of $E$ denoted $\overline{E}$ is 
        $$\overline{E} := E \cup E'$$
    \end{enumerate}
\end{definition}

\paragraph{Review.} Fix $(X,d)$ a metric space. 
\begin{enumerate}
    \item $N_\delta(x_0)$ is open $\forall \delta > 0, \ \forall x_0 \in X$. 
    \item $E$ is open iff $E^c$ is closed. 
    \item $\overline{E}$ is closed, in fact $\overline{E}$ is the smallest closed set containing $E$. 
    \item The union of any collection of open sets is open. 
    \item A finite intersection of a collection of open sets is open. 
    \item Intersection of any collection of closed sets is closed. 
    \item Finite union of closed is closed.  
    \item If $x \in E'$ then $N_\delta(x) \cap E'$ is an infinite set. 
\end{enumerate} 

\begin{definition}
    Let $x_n$ be a sequence in $X$, we say that $x_n$ converges to $x_0 \in X$ iff $\forall \varepsilon > 0$ $\exists N \in \N$ s.t 
    $$d(x_n, x_0) < \varepsilon \quad \forall n \geq N$$  
\end{definition}

\begin{enumerate}
    \item The limit of a sequence (if exists) is unique. 
    \item $x_n$ converges to $x_0$ in $(X,d)$ iff $d(x_n, x_0)$ converges $0$ in $(\R, \norm{})$. 
\end{enumerate}

\begin{example}
    $x_n = (1/n + 1, e^{-n})$ in $\R^2$. Find the limit of $x_n$ in $(\R^2, \norm{.}_{\infty})$, $(\R^2, \norm{.}_1)$, $(\R^2, \norm{.}_2)$. 
    \begin{itemize}
        \item $$\norm{(1/n + 1, e^{-n}, (1,0))}_{\infty} = \max (1/n, e^{-n}) \leq 1/n + e^{-n} \to 0$$
        \item $x_0 = (1, 0)$ 
        $$\norm{(1/n + 1, e^{-n}, - (1, 0))}_1 = \norm{(1/n, e^{-n})}_1 = |1/n| + |e^{-n}| \to 0$$

        \item $$\norm{(1/n + 1, e^{-n}) - (1,0)}_2 = \sqrt{1/n^2 + e^{-2n}} \to 0$$
    \end{itemize}
\end{example}

\begin{remark}
    $x_n \to x_0$ in $(\R^k, \norm{.}_\infty)$ 
    $$(x_{1,n}, x_{2,n}, \cdots, x_{k,n}) \to (x{1,0}, \cdots, x_{k,0})$$
    Iff 
    \begin{align*}
        x_{1,n} &\to x_{1,0} \\
        x_{2,n} &\to x_{2,0} \\
        &\vdots \\
        x_{k,n} &\to x_{k,0} 
    \end{align*} 
    in $(\R, \norm{.})$. 
\end{remark}

\begin{proof}
    $|x_{i, n} - x_{i,0} \leq \norm{x_n - x_0}_\infty \leq |x_{1,n} - x_{1,0}| + \cdots + |x_{k,n} - x_{k,0}|$
    Therefore $x_n \to x_0$ in $(\R^k, \norm{.}_\infty)$ $\iff$ $\norm{x_n - x_0}_\infty \to 0$ and $\iff$ $|x_{i,n} - x_{i,0}| \to 0$ $\forall i$ and hence 
    $\iff$ $x_{i,n} \to x_{i,0}$ $\forall i$. 
\end{proof}

\begin{remark}
    Take $x \in \R^n$, then 
    $$\norm{x}_\infty = \max_i (x_i) \leq |x_1| + \cdots + |x_n| = \norm{x}_1 \leq n \norm{x}_\infty$$
    Therefore 
    $$ \norm{x}_\infty \leq \norm{x}_1 \leq n \cdot \norm{x}_\infty$$
    Similarly, show something for $\norm{x}_2$ and $\norm{x}_\infty$. 
\end{remark}

\begin{definition}
    Given a vector space $V$ and norms $\norm{.}_a$ and $\norm{.}_b$ we say that $\norm{.}_a$ and $\norm{.}_b$ are 
    equivalent iff $\exists c_1, c_2 > 0 \in \R$ s.t $\forall x \in V$
    $$c_1 \norm{x}_b \leq \norm{.}_a \leq c_2 \norm{x}_b$$
\end{definition}

\begin{lemma}
    $(X,d)$ metric space and $E \subseteq X$, then $E$ is closed iff for sequence $x_n \in E$ that converges 
    $\lim_{n \to \infty} x_n \in E$. 
\end{lemma}

\begin{proof}
    \begin{itemize}
        \item[$\implies$] Assume that there exists a sequence $x_n$ in $E$ that converges to $x_0$ s.t $x_0 \not\in E$, 
        take $\varepsilon > 0$, then since $x_n \to x_0$ then $\exists N \in \N$ s.t 
        $$x_n \in N_\varepsilon (x_0) \quad \forall n \geq N$$
        But $x_n \in E$, 
        $$N_\varepsilon^*(x_0) \cap E \neq \varnothing$$
        Therefore 
        $$x_0 \in E'$$
        Contradiction to the fact that $E$ is closed. 

        \item Take $x_0 \in E'$, let $\varepsilon_n = 1/n$ then 
        $$\exists x_n \in N_n^* (x_0) \cap E$$
        Consider $\set{x_n}_{n \in \N}$ $x_n$ is a sequence in $E$ s.t 
        $$d(x_n, x_0) < 1/n \implies x_n \to x_0$$
        and hence $x_0 \in E$. 
    \end{itemize}
\end{proof}

\begin{theorem}
    Given a vector space $V$ and norms $\norm{.}_a$ and $\norm{.}_b$ then the following are all equivalent: 
    \begin{enumerate}
        \item $\norm{.}_a$ and $\norm{.}_b$ are equivalent ($c_1 \norm{x}_b \leq \norm{x}_a \leq c_2 \norm{x}_b$). 
        \item $x_n \to x_0$ in $\norm{.}_a$ $\iff$ $x_n \to x_0$ in $\norm{.}_b$. 
        \item $E$ is closed in $(V, \norm{.}_a)$ $\iff$ $E$ is closed in $(V, \norm{.}_b)$. 
        \item $U$ is open in $(V, \norm{.}_a)$ $\iff$ $U$ is open in $(V, \norm{.}_b)$. 
    \end{enumerate}
\end{theorem}

\begin{proof}
    \begin{itemize}
        \item[$1 \implies 2$] $$c_1 \norm{x_n - x_0}_b \leq \norm{x_n - x_0}_a \leq c_2 \norm{x_n - x_0} \leq \frac{c_2}{c_1} \norm{x_n - x_0}_a$$
        Apply Squeeze. 
        \item[$2 \implies 3$] By the Lemma. 
        \item[$3 \implies 4$] Trivial. 
        \item[$4 \implies 1$] $B_{\norm{.}_a} (0, 1)$ open in $(V, \norm{.}_a)$ therefore it is open in $(V, \norm{.}_b)$. 
        Hence $0$ is an interior point in $(V, \norm{.}_b)$. Hence $\exists \delta > 0$ 
        $$B_{\norm{.}_b} (0, \delta) \subseteq B_{\norm{.}_a} (0, 1)$$
        Let $x \in V$, $x \neq 0$. 
        $$\frac{\delta}{2} \cdot \frac{x}{\norm{x}_b} \in B_{\norm{.}_b}(0, \delta) \subseteq B_{\norm{.}_a} (0,1)$$
        Therefore 
        $$\norm{\frac{\delta}{a} \frac{x}{\norm{x}_b}}_a < 1$$
        We get 
        \begin{align*}
            \frac{\delta}{2 \norm{x}_b} \norm{x_a} < 1 &\implies \norm{x}_a \leq \frac{2}{\delta} \norm{x}_b \ \forall x \in V
        \end{align*}
    \end{itemize}
\end{proof}

\subsection{Compact Sets} 
\begin{definition}
    $(X, d)$ metric space, $K \subseteq X$. We say that $K$ is \emph{compact} iff for every open cover 
    $$\mathcal{G} = \set{G_\alpha}_{\alpha \in I}$$
    of $K$ has a finite subcover. 
\end{definition}

\begin{example}
    $(0, 1]$ is not compact in $(R, \abs{.})$, take 
    $$\mathcal{G} = \set{(1/n, 10)}_{n \in \N}$$
    which has no finite subcover. 
\end{example}

\begin{theorem}[Heine-Borel]
    In $\R$ $[a,b]$ is compact.
\end{theorem}

\begin{example}
    If $X$ is finite, then $X$ is compact. 
\end{example}

\begin{proof}
    Let $X = \set{x_1, \cdots, x_n}$ and let 
    $$\mathcal{G} = \set{G_\alpha}_{\alpha \in I}$$
    be an open cover $X$. Then for every $x_i$, fix some $G_{\alpha_i}$ s.t 
    $$x = \set{x_1, \cdots, x_n} \subseteq \bigcup_{i=1}^n G_{\alpha_i}$$
\end{proof}

\begin{proposition}
    $(X, d)$ metric space and $K \subseteq X$ then every closed subset $E$ of $K$ is compact. 
\end{proposition}

\begin{proof}
    Let $\mathcal{G} = \set{G_\alpha}_{\alpha \in I}$ be an open cover of $E$, then 
    $$E \subseteq \bigcup_{\alpha \in I} G_\alpha$$
    Therefore 
    $$K \subseteq \bigcup G_\alpha \cup E^c, \quad \text{$E^c$ open}$$
    Therefore 
    $$\mathcal{G'} = \mathcal{G} \cup \set{E^c} \quad \text{is an open cover of $K$}$$
    Therefore it has a finite subcover and 
    $$K \subseteq G_{\alpha_1} \cup \cdots \cup G_{\alpha_k} \cup E^c$$
    Hence 
    $$E \subseteq G_{\alpha_1} \cup \cdots \cup G_{\alpha_k}$$
    Which is a finite subcover of $\mathcal{G}$. 
\end{proof}

\begin{definition}
    $(X,d)$ metric space and $x_n$ sequence in $X$ and 
    $$\varphi \colon \N \to \N \quad \text{strictly increasing}$$
    We say that $b_n = a_{\varphi(n)}$ is a \emph{subsequence} of $a_n$. 
\end{definition}

\begin{proposition}
    $x_n$ converges to $x_0$ $\iff$ every subsequence $x_{n_k}$ converges to $x_0$. 
\end{proposition}

\begin{definition}
    $(X,d)$ metric space and $K \subseteq X$. We say that $K$ is sequentially compact iff every sequence $x_n \in K$ has a convergent subsequence in $K$. 
\end{definition}

\begin{proposition}
    $(X,d)$ metric space, $K$ sequentially compact iff every infinite $E \subseteq K$ has a limit point in $K$. 
\end{proposition}

\begin{proof}
    \begin{itemize}
        \item[$\implies$] Let $K$ be sequentially compact, take $E \subseteq K$ an infinite set. We can extract from $E$ a sequence of 
        distinct elements $x_n \in E \subseteq K$. Therefore $x_n$ has a convergence subsequence $x_{n_k}$ s.t the limit of $x_{n_k} = x_0 \in K$. 
        Moreover, $x_{n_k}$ are distinct in $E$ therefore $x_0 \in E' \cap K$.
        
        \item[$\impliedby$] Let $x_n$ be a sequence in $K$, take 
        $$E = \set{x_n}_{n \in \N}$$ 
        We have two cases: 
        \begin{itemize}
            \item \textbf{$E$ is finite.} Then $\exists x_0$ and $\exists n_1 < n_2 < \cdots$ (infinitely many) s.t 
            $$x_{n_k} = x_0, \quad \forall k \in \N$$
            Therefore 
            $$\lim_{k \to \infty} x_{n_k} = x_0 \in E$$
            We found a converging subsequence. 

            \item \textbf{$E$ is infinite.} Then by the assumption, we have that $E$ has a limit point $x_0$ in $K$. 
            \begin{itemize}
                \item[$\varepsilon = 1$] $\exists n_1$ s.t $x_{n_1} \in N_1^*(x_0) \cap E$. 
                \item[$\varepsilon = 1/2$] $\exists n_2 > n_1$ s.t $x_{n_2} \in N_{1/2}^*(x_0) \cap E$ (since $N_{1/2}^*(x_0) \cap E$ is infinite).  
            \end{itemize} 
            we can construct $n_k > n_{k-1}$ s.t $x_{n_k} \in N^*_{1/k} (x_0) \cap E$, therefore 
            $$\lim_{n \to \infty} x_{n_k} = x_0 \in K$$
        \end{itemize}
    \end{itemize}
\end{proof}

\begin{example}
    In $(\Q, \abs{.})$, $K = \Q \cap [0,1]$ is closed and bounded. $P_n$ increasing sequence in $\Q \cap [0,1]$ s.t 
    $$P_n \to 1/\sqrt{2} \in \R$$
    Consider 
    $$\mathcal{G} = \set{(-1, P_1) \cap \Q, (-1, P_2) \cap \Q, \cdots, (-1, P_n) \cap \Q, \cdots, (1/\sqrt{2}, 10) \cap \Q}$$
\end{example}

\begin{proposition}
    $K \subseteq X$ is compact $K$ is sequentially compact.     
\end{proposition}

\begin{proof}
    \begin{itemize}
        \item Let $E \subseteq K$ infinite, suppose that $E' \cap K = \varnothing$. Take $x \in K \implies x \not\in E'$, 
        therefore $\exists \delta_x > 0$ s.t 
        $$N_{\delta_x}^*(x) \cap E = \varnothing$$
        Let $\mathcal{G} = \set{N_{\delta_x}(x)}_{x \in K}$, is an open cover of $K$ compact, hence it has a finite subcover 
        $$K \subseteq N_{\delta_{x_1}} (x_1) \cup \cdots \cup N_{\delta_{x_1}}(x_k)$$
        But $E \subset K \implies E = (E \cap K)$ hence 
        \begin{align*}
            E &\subseteq (N_{\delta_{x_1}} (x_1) \cap E) \cup \cdots \cup (N_{\delta_{x_1}}(x_k) \cap E)\\
            &\subseteq \set{x_1, \cdots, x_k}
        \end{align*}
        therefore $E$ is finite, contradiction.
    \end{itemize}
\end{proof}

\subsection{Complete sets}
\begin{definition}
    Given a sequence $x_n$, we say that $x_n$ is Cauchy iff for every $\varepsilon > 0$ there exists $N \in \N$ s.t 
    $$d(x_n, x_m) < \varepsilon \quad \text{for $n, m \geq N$}$$
\end{definition}

\begin{proposition}
    If $x_n$ converges, then $x_n$ is Cauchy. 
\end{proposition}

\begin{proof}
    Triangle inequality.
\end{proof}

\begin{remark}
    The converse is false. In $(\Q, \abs{.})$, in $(\R, \abs{.})$ there exists some sequence $x_n \in \Q$ s.t 
    $$x_n \to \sqrt{2} \quad \text{as } n \to \infty$$
    But $x_n \to \sqrt{2}$ implies that $x_n$ is Cauchy in $(\R, \abs{.})$, therefore $x_n$ is Cauchy in $\Q$. 
    Suppose $x_n$ converges to $P$ in $(\Q, \abs{.})$, therefore $x_n \to p \in (\R, \abs{.})$ by uniqueness of limit, 
    $p = \sqrt{2}$, contradiction. 
\end{remark}

\begin{definition}
    $(X, d)$ metric space and $K \subseteq X$, we say that $K$ is complete iff every Cauchy Sequence in $K$ converges in $K$.
\end{definition}

\begin{example}
    $(\R, \abs{.})$ is complete (Math $210$)
\end{example}

\begin{proposition}
    If $K$ is complete then $K$ is closed. 
\end{proposition}

\begin{proof}
    Let $x_n$ be a sequence in $K$ s.t $x_n \to x_0 \in X$. Since $x_n$ converges, then $x_n$ is Cauchy and $K$ is complete then 
    $x_n$ converges in $K$, therefore $x_0 \in K$.      
\end{proof}

\begin{proposition}
    $K$ complete and $E \subseteq K$ is closed, then $E$ is complete. 
\end{proposition}

\begin{proof}
    Let $x_n$ be a Cauchy sequence in $E$, then $x_n$ is Cauchy in $K$ which is complete, therefore $x_n \to x_0 \in K$. 
    But since $E$ is closed, we get that $x_0 \in E$. 
\end{proof}

\begin{proposition}
    $(X, d)$ metric space and $x_n$ Cauchy sequence in $K \subseteq X$. If $x_n$ has a convergent subsequence $x_{n_k}$ then $x_n$ converges to 
    the subsequential limit.  
\end{proposition}

\begin{proof}
    Triangle inequality. 
\end{proof}

\begin{proposition}
    $(X,d)$ metric space and $K \subseteq X$ is sequentially compact then $K$ is complete (and therefore closed). 
\end{proposition}

\begin{proof}
    Let $\set{x_n}$ be a Cauchy sequence in $K$, $K$ is sequentially compact therefore $x_n$ has a convergent subsequence $x_{n_k}$ that converges in $K$. 
    Therefore by the previous proposition we get that $x_n$ converges in $K$. 
\end{proof}

\begin{theorem}
    Let $V$ be a finite dimensional inner product space over $\R$, then $V$ is complete w.r.t norm induced by the inner product. Therefore 
    $K\subseteq V$ is complete $\iff$ $K$ is closed. 
    
    (For the converse $K \subseteq V$ which is closed, then $K$ is complete).
\end{theorem}

\begin{proof}
    Let $S = \set{v_1, \cdots, v_l}$ be an orthonormal basis of $V$, take $x_n$ be a cauchy sequence in $V$. Therefore 
    $$x_n = r_{1,l} v_1 + \cdots + r_{l,n} v_l$$
    Take $r_{1,n} = \angles{x_n, v_1}$ therefore 
    \begin{align*}
        \abs{r_{1,m} - r_{1,n}} &= |\angles{x_{m}, v_1} -\angles{x_n, v_1}| \\
        &= \abs{\angles{x_m - x_n, v_1}} \\
        &\leq \norm{x_m - x_n} \norm{v_1} \quad \text{By C.S} \\ 
    \end{align*}    
    Let $\varepsilon > 0$ then $\exists N \in \N$ s.t 
    $$\norm{x_m - x_n} < \varepsilon \implies r_{1,n} \text{ is Cauchy in $\R$}$$
    THerefore $r_{1,n}$ converges to $r_1 \in \R$. Do the same fore very coordinate, and define 
    $$x = r_1 v_1 + \cdots + r_n v_n$$
    Therefore 
    \begin{align*}
        \norm{x_n - x} &= \norm{(r_{1,n} - r_1) v_1 + \cdots + (r_{1,l} - r_l) v_l} \\
        &\leq \norm{r_{1,n} - r_1} \norm{v_1} + \cdots + \norm{r_{l,n} - r_l} \norm{v_l}
    \end{align*}
    Therefore 
    $$x_n \to x$$
\end{proof}

\subsection{Total Boundedness}

\begin{definition}
    $(X, d)$ metric space and $K \subseteq X$ we say that $K$ is totally bounded iff for every $\varepsilon > 0$ $\exists x_1, \cdots, x_t \in K$ s.t 
    $$K \subseteq N_{\varepsilon} (x_1) \cup \cdots \cup N_\varepsilon (x_t)$$ 
\end{definition}

\begin{definition}
    $K \subseteq X$ is bounded iff $\exists R > 0$ s.t 
    $$K \subseteq N_R (x_0) \quad \text{for some $x_0 \in K$}$$
    We say that a sequence is bounded iff 
    $$E = \set{x_n}_{n \in \N} \quad \text{is bounded}$$
\end{definition}

\begin{proposition}
    Cauchy sequences are bounded.
\end{proposition}

\begin{proposition}
    $K \subseteq X$ is totally bounded it is bounded. 
\end{proposition}

Note that the converse is false. 
\begin{example}
    $(\R, d_{disc})$, $\R$ is bounded since 
    $$\R \subseteq N_2(0)$$
    but it isn't totally bounded, take $\varepsilon = 1/2$ then if it is totally bounded, then $\exists x_1 \cdots x_t \in \R$ s.t 
    $$\R \subseteq N_{1/2} (x_1) \cup \cdots \cup N_{1/2} (x_t) = \set{x_1, \cdots, x_t}$$
\end{example}

\begin{proof}
    Take $\varepsilon = 1$ then $K \subseteq N_1(x_1) \cup N_1 (x_t)$ for $x_1, \cdots, x_t \in K$. 
    Take $x_1$ and let $d = \max_{k \in \set{1, \cdots, t}} (x_1, x_t)$, therefore for any $x \in K$ we have that 
    $$d(x, x_1) \leq d(x, x_i) + d(x_i, x_1)$$
    for some $x_i$ s.t $x \in N_1(x_i)$ (Exists by the above), therefore 
    $$d(x, x_1) \leq 1 + d \implies x \in N_{1 + d} (x_1)$$
\end{proof}

\begin{proposition}
    $K$ is sequentially compact then $K$ is totally bounded.
\end{proposition}

\begin{proof}
    Suppose not, then $\exists \varepsilon_0$ s.t $K$ cannot be covered by finitely many $\varepsilon_0$-neighborhoods. Fix some $x_1$ then 
    $\exists x_2 \in K \setminus N_{\varepsilon_0} (x_1)$, Similarly there exists some $x_3 \in K \setminus N_{\varepsilon_0} (x_1) \cup N_{\varepsilon_0} (x_2)$ 
    and in general 
    $$\exists x_n \in K \setminus ( N_{\varepsilon_0} (x_1) \cup \cdots \cup  N_{\varepsilon_0} (x_{n-1}))$$
    Therefore we have that for $n \neq m$
    $$d(x_n, x_m) \geq \varepsilon_0$$
    Therefore $x_n$ doesn't have a convergent subsequence; contradiction. 
\end{proof}

\begin{lemma}
    Every subset of a totally bounded set is totally bounded. 
\end{lemma}

\begin{proof}
    $K \subseteq X$ totally bounded, and $E \subseteq K$; take $\varepsilon > 0$, therefore $\exists x_1, x_2, \cdots, x_t \in K$ s.t 
    $$K \subseteq N_{\varepsilon/2}(x_1) \cup \cdots \cup N_{\varepsilon/2} (x_t)$$
    Ignore the neighborhood that don't intersect $E$, renumbering we have $x_1, \cdots, x_l$ s.t 
    $$N_{\varepsilon/2} (x_i) \cap E \neq \varnothing$$
    We get 
    $$E \subseteq N_{\varepsilon/2} (x_1) \cup \cdots \cup N_{\varepsilon/2} (x_l)$$
    Take $y_i \in N_{\varepsilon/2} (x_i) \cap E$, then 
    $$N_{\varepsilon} (y_i) \supseteq N_{\varepsilon/2} (x_i)$$
    Therefore $E \subseteq N_{\varepsilon}(y_1) \cup \cdots \cup N_{\varepsilon} (y_l)$ and hence $E$ is totally bounded. 
\end{proof}

\begin{example}
    $[0,1]$ is totally bounded (since it is compact), therefore $(0,1)$ is totally bounded.  
\end{example}

\begin{proposition}
    $(X,d)$ Metric space, $K \subseteq X$ is complete and totally bounded then $K$ is compact. 
\end{proposition}

\begin{proof}
    Assume $K$ is not compact, fix $\mathcal{G} = \set{G_\alpha}_{\alpha \in I}$ an open cover of $K$ with no finite subcover. 
    Let $\varepsilon = 1$, $K$ is totally bounded then $K$ can be covered by finitely many $1$-balls with center in $K$. Hence $\exists x_1 \in K$ s.t 
    $K_1 = N_1 (x_1) \cap K$ cannot be covered by finitely many open sets in $\mathcal{G}$. But $K_1$ is totally bounded (subset of $K$), 
    now let $\varepsilon = 1/2$, $\exists x_2 \in K_1$ s.t $K_2 = N_{1/2} (x_2) \cap K_1$ cannot be covered by finitely many open sets in $\mathcal{G}$. 
    Similarly, we construct a sequence $x_n$ and 
    $$K_n = N_{1/n} (x_{n}) \cap K_{n-1} \quad x_n \in K_{n-1}$$
    such that $K_n$ cannot be covered by finitely many $G_\alpha$'s. Notice that 
    $$K_0 = K \supseteq K_1 \supseteq K_2 \supseteq \cdots$$
    \textbf{Claim.} $x_n$ is Cauchy. Indeed, let $\varepsilon > 0$ then $\exists N$ s.t 
    $$1/N < \varepsilon / 2$$
    Now take $m, n > N$, then 
    $$x_n \in K_{n-1} \subseteq K_N, \quad x_m \in K_{m-1} \subseteq K_N$$
    therefore $x_n, x_m \in K_N = N_{1/N} (x_N) \cap K_{n-1}$, and hence 
    $$d(x_n, x_m) \leq 2/N < \varepsilon$$
    By completeness of $K$, we get that $x_n$ converges to some $x_0 \in K$; since $\mathcal{G} = \set{G_\alpha}$ is an open cover of $K$, 
    there exists $\alpha_0 \in I$ s.t $x_0 \in G_{\alpha_0}$ (open set). So $\exists \varepsilon_0$ s.t 
    $$N_{\varepsilon_0} (x_0) \subseteq G_{\alpha_0}$$
    But $x_n \to x_0$ then $\exists N_1$ s.t 
    $$x_N \in N_{\varepsilon_0/2} (x_0) \text{ and } 1/N < \varepsilon_0 /2$$
    Therefore 
    $$K_N \subseteq N_{1/N} (x_n) \subseteq N_{\varepsilon_0} (x_0) \subseteq G_{\alpha_0}$$
    contradiction.
\end{proof}

\begin{theorem}
    $V$ finite dimensional product normed space (over $\R$). Let 
    $S = \set{v_1, \cdots, v_n}$ a basis of $V$. Then the cube 
    $$Q_S = \set{r_1 v_1 + r_2 v_2 + \cdots + r_n v_n \colon 0 \le r_i \le 1}$$
    is compact. 
\end{theorem}

\begin{proof}
    Let $x^k = r_{1,k} v_1 + r_{2,k} v_2 + \cdots + r_{n,k} v_n$ a sequence in $Q_S$, now clearly 
    $$r_{1,k} \text{ is a sequence in $[0,1]$, compact set}$$
    Therefore $r_{1,k}$ has a convergent subsequence $r_{1,k_n}$. Similarly $r_{2, k_l} \in [0,1]$ therefore 
    $r_{2,k_l}$ has a convergent subsequence $r_{2,k_{l_t}}$. Doing this $n$ times we get that $\exists$ a subsequence 
    $\set{k_j}$ s.t 
    $$r_{1,k_j}, \ r_{2,k_j}, \cdots, \ r_{n,k_j} \to r_1, r_2, \cdots, r_n$$ 
    take 
    $$x_0 = r_1 v_1 + \cdots + r_n v_n$$
    We can show that 
    $$\norm{x_0 - x^{k_j}} \le |r_{1,k_j} - r_1| \norm{v_1} + \cdots + |r_{n,k_j} - r_n| \norm{v_n} \to 0$$
    then 
    $$x^{k_j} \to 0$$ 
    Hence $Q_S$ is (sequentially) compact. 
\end{proof}

\begin{theorem}
    Let $V$ be a finite inner product space then 
    $$K \subseteq V \text{ is compact} \iff K \text{ closed and bounded}$$ 
\end{theorem}

\begin{proof}
    \begin{itemize}
        \item $K$ compact $\implies$ complete $\implies$ closed and $\implies$ totally bounded $\implies$ bounded.
        \item $K$ is bounded, therefore $\exists R > 0$ s.t $K \subseteq B_{\norm{.}} (0, R)$. Let $S = \set{v_1, \cdots, v_n}$ be a basis of $V$ and 
        Take $x \in K$ therefore 
        $$x = r_1 v_1 + \cdots + r_n v_n$$
        then $r_i = \norm{\angles{x, v_i}} \le \norm{x} \cdot \norm{v_i}$ 
        $$B_{\norm{.}} (0, R) \subseteq \set{r_1 v_1 + \cdots + r_n v_n \colon r_i \le R}$$
        Therefore $K \subseteq$ some cube and hence $K$ is compact.  
    \end{itemize} 
\end{proof}

\begin{proposition}
    In a finite dimensional inner product space we get that bounded $\iff$ totally bounded. 
\end{proposition}

\begin{proof}
    \begin{itemize}
        \item[$\implies$] Bounded $\implies$ can be put in a cube (compact) $\implies$ Totally bounded. 
        \item[$\impliedby$] True in general.
    \end{itemize}
\end{proof}

\subsection{Continuity}
\begin{definition}
    $X, Y$ metric spaces and $E \subseteq X$ and $f \colon E \to Y$ map; $x_0 \in E'$ we say that 
    $$\lim_{x \to x_0} f(x) = y \in Y$$
    iff for every $\varepsilon > 0$ $\exists \delta > 0$ s.t 
    $$f(x) \in N_\varepsilon(y_0) \text{ for } x \in N_\delta^*(x_0) \cap E$$
\end{definition}

\begin{definition}
    $f \colon E \to Y$ continuous at $x_0$ iff for every $\varepsilon > 0$, $\exists \delta > 0$ s.t $f(x) \in N_{\varepsilon}(x_0)$ for 
    $x \in N_\delta(x_0) \cap E$; i.e 
    $$d_Y(f(x), f(x_0)) < \varepsilon \quad \text{for } 0 < d_X(x, x_0) < \delta$$
\end{definition}

\begin{remark}
    $f$ is continuous at $x_0$ $\iff$ $x_0$ is an isolated point or $x_0 \in E'$ and $\lim{x \to x_0} f(x) = f(x_0)$. 
\end{remark}


\begin{proposition}
    $\lim_{x \to x_0} = y_0$ iff for every sequence $x_n \to x_0$ we have that $f(x_n) \to y_0$.
    Therefore $f$ is continuous at $x_0$ iff for every $x_n \to x_0$ we have that $f(x_n) \to x_0$. 
\end{proposition}

\begin{corollary}
    if $f, g$ are continuous then $f + g, \ cf, \ f \cdot g, \ f / g$ are all continuous at $x_0$.
\end{corollary}


\end{document}